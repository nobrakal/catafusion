\documentclass[format=sigplan]{acmart}

\usepackage[utf8]{inputenc}
\usepackage[english]{babel}
\usepackage{amsmath}
\usepackage{amssymb}

\usepackage{minted}
\newcommand{\minline}[1]{\mintinline{Haskell}{#1}}

%
% defining the \BibTeX command - from Oren Patashnik's original BibTeX documentation.
\def\BibTeX{{\rm B\kern-.05em{\sc i\kern-.025em b}\kern-.08emT\kern-.1667em\lower.7ex\hbox{E}\kern-.125emX}}

% Rights management information.
% This information is sent to you when you complete the rights form.
% These commands have SAMPLE values in them; it is your responsibility as an author to replace
% the commands and values with those provided to you when you complete the rights form.
%
% These commands are for a PROCEEDINGS abstract or paper.
\copyrightyear{2019}
\acmYear{2019}
%\setcopyright{acmlicensed}
\acmConference[]{}{}{}
%\acmBooktitle{Woodstock '18: ACM Symposium on Neural Gaze Detection, June 03--05, 2018, Woodstock, NY}
%\acmPrice{15.00}
%\acmDOI{10.1145/1122445.1122456}
%\acmISBN{978-1-4503-9999-9/18/06}

\begin{document}
\title{Fusion}
\author{Alexandre Moine}
\email{alexandre@moine.me}

\affiliation{%
	\institution{Université Paris Diderot}
	\streetaddress{5 Rue Thomas Mann}
	\city{Paris}
	\country{France}
	\postcode{75013}
}

\begin{abstract}
Recursive structure are everywhere in Haskell, ...  . We therefore propose a simple automatic technique to perform deforestation on recursive types using the rewrite rules system.
\end{abstract}

%
% The code below is generated by the tool at http://dl.acm.org/ccs.cfm.
% Please copy and paste the code instead of the example below.
%
\begin{CCSXML}
TODO
\end{CCSXML}

%% \ccsdesc[500]{Computer systems organization~Embedded systems}
%% \ccsdesc[300]{Computer systems organization~Redundancy}
%% \ccsdesc{Computer systems organization~Robotics}
%% \ccsdesc[100]{Networks~Network reliability}

%
% Keywords. The author(s) should pick words that accurately describe the work being
% presented. Separate the keywords with commas.

\keywords{Haskell, GHC, deforestation}

%
% This command processes the author and affiliation and title information and builds
% the first part of the formatted document.
\maketitle

\section{Introduction}

A Haskell program can be seen as the composition of many other small programs. It is often the right way to see software development but can lead to optimization problems. For example the expression:
\begin{minted}{Haskell}
foldr (+) 0 (map (* 2) xs)
\end{minted}
is equivalent to
\begin{minted}{Haskell}
foldr ((+) . (*) 2) 0 xs
\end{minted}

\noindent While the first one is more comprehensible, it builds an intermediate list that is not needed. The second version optimize this and will only go through the original list once.
This optimization, called \emph{deforestation} (Wadler \cite{WADLER1990231}) can be made at compile-time and thus allows the user to write a readable code while the compiler optimize it when it can; this is known for a long time, explained by Gill et al. \cite{Gill:1993:SCD:165180.165214} and implemented by Peyton Jones et al. \cite{playing-by-the-rules-rewriting-as-a-practical-optimisation-technique-in-ghc} for GHC.

The purpose of this paper is to generalize this work for fixed-point recursive types with a minimalistic library that allows programmers to use deforestation simply. We first explain the generalization of "fold" functions in section \ref{sec:recschemes}, then explore the classical deforestation on recursive types in section \ref{sec:rectypes} and finally explore a generalization for "generalized recursive functions" while keeping compatibility with the previous work in section \ref{sec:para}.

\section{Recursion schemes}
\label{sec:recschemes}
%!TEX root = fusion.tex

Recursion schemes are almost everywhere in functional programming. They were popularized by Meijer et al. \cite{4cec4a43c86444479dc0003182424795}.

\subsection{Catamorphisms and paramorphisms}
A \emph{catamorphisms} generalize a fold on a recursive structure. For example, the catamorphism of lists is the so-called \minline{foldr}.
\begin{minted}{Haskell}
foldr :: (a -> b -> b) -> b -> [a] -> b
foldr k z = go
  where
    go []     = z
    go (y:ys) = y `k` go ys

\end{minted}
TODO

A \emph{paramorphism} allow the combination function to have a look on the unconsummed part of the structure (imitating the full power of recursion). It can be defined for lists as

\begin{minted}{Haskell}
para :: (a -> [a] -> b -> b) -> b -> [a] -> b
para k z = go
  where
    go []     = z
    go (y:ys) = k y ys (go ys)
\end{minted}

TODO equivalence para/foldr

\subsection{Anamorphisms and apomorphisms}
TODO

\subsection{Fusion of recursive scheme - hylomorphisms}
TODO

\subsection{Deforestation}
Deforestation is a process first introduced by Wadler \cite{WADLER1990231} and realized in Haskell by Gill et al. \cite{Gill:1993:SCD:165180.165214}. It is the idea that one can eliminate a structure produced by an anamorphism and destructed right away by a catamorphism.

The first example of this concept is the expression:
\begin{minted}{Haskell}
foldr c e (map f xs)
\end{minted}
\minline{map} will destruct the original list and produce a new one, which will be destructed after with \minline{foldr}.

Obviously, the above expression is equivalent to a more efficient one:
\begin{minted}{Haskell}
foldr (c . f) e xs
\end{minted}
The above expression will go through the list only once.

Note that the rewriting operation we made in this example is not at all dependent of the list structure: we just used the fact that we know how is made the list produced by \minline{map}. This concept can be simply generalized to recursive structure, as we will show in the next section.


\section{Deforestation on recursive types}
\label{sec:rectypes}
%!TEX root = fusion.tex

Deforestation is a process first introduced by Wadler \cite{WADLER1990231} and realized in Haskell by Gill et al. \cite{Gill:1993:SCD:165180.165214}. It is the idea that one can eliminate a structure produced by "good function" (like an anamorphism) and destructed right away by a catamorphism.

The first example of this concept is the expression:
\begin{minted}{Haskell}
foldr c e (map f xs)
\end{minted}
\minline{map} will destruct the original list and produce a new one, which will be destructed after with \minline{foldr}. The key thing is that we know how \minline{map} will construct the new list. Obviously, the above expression is equivalent to a more efficient one:
\begin{minted}{Haskell}
foldr (c . f) e xs
\end{minted}
The above expression will go through the list only once.

Note that the rewriting operation we made in this example is not at all dependent of the list structure: we just used the fact that we know how is made the list produced by \minline{map}. This concept can be simply generalized to recursive structure, as we will show in the next section.


\subsection{The idea}
Deforestation can be implemented on recursive types imitating the work of Gill et al. \cite{Gill:1993:SCD:165180.165214}.

Applying \minline{cata go} to a recursive structure has just the effect to replace every occurrence of \minline{Fix} by \minline{go} in the structure. Consequently, the idea is to abstract structure-producing functions with respect to \minline{Fix}. This is done using \minline{buildR}.

\begin{minted}{Haskell}
type Cata f = forall b. (f b -> b) -> b

buildR :: Cata f -> Fix f
buildR g = g Fix
\end{minted}

\noindent Intuitively, the argument of \minline{buildR} is a function that can, for all type \minline{b}, and given something to "perform recursion" (that is, produce a \minline{b} from a \minline{(f b)}) can produce something of type \minline{b}. For example, you can use \minline{buildR} to define \minline{ana}:
\begin{minted}{Haskell}
ana :: Functor f => (b -> f b) -> b -> Fix f
ana f b = buildR
  (\comb -> let c =  comb . fmap c . f in c b)
\end{minted}

\subsubsection{Validity}
Now, we have the lovely property that:
$$cata\ go\ (buildR\ g)\ =\ g\ go$$

This is exactly what we want: the right-side of the equation \emph{does not} build an intermediate structure. The validity of the above rule is guaranteed by the type (more precisely the rank-2 one).

\begin{theorem}
Using, for a fixed parametric type $F$, a valid instance of $Functor$, and a type $A$:
\begin{itemize}
	\item $g :: \forall\ B.\ (F\ B \to B) \to B$
	\item $go :: F\ A \to A $
\end{itemize}
$$cata\ go\ (buildR\ g)\ =\ g\ go$$
\end{theorem}
\begin{proof}
This is a consequence of free theorems of Wadler \cite{Wadler:1989:TF:99370.99404}, as for the \verb|fold/build| rule \cite{Gill:1993:SCD:165180.165214}. The free theorem associated with $g$'s type is that, for all types $C$ and $D$, $f\ ::\ C \to D$, $p\ ::\ F\ C \to C$ and $q\ ::\ F\ D \to D$, the we have the following implication:
\begin{align*}
[\ \forall& x::C.\ f\ (p\ x)\ =\ q\ (fmap\ f\ x)\ ]\\
&\implies f\ (g\ p)\ =\ g\ q
\end{align*}
Now we can pose:
\begin{itemize}
	\item $C\ =\ Fix\ F$, $D\ =\ A$
	\item $f\ =\ cata\ go$
	\item $p\ =\ fix$
	\item $q\ =\ go$
\end{itemize}
then we have
\begin{align*}
&[\ \forall x::F\ (Fix\ F).\\
&cata\ go\ (fix\ x)\ =\ go\ (fmap\ (cata\ go)\ x)\ ]\\
&\implies cata\ go\ (g\ fix)\ =\ g\ go
\end{align*}
Given the definition of $cata$, the premise trivially holds. So we have the conclusion:
$$cata\ go\ (g\ fix)\ =\ g\ go$$
Given the definition of $buildR$, this is exactly what we required.
\end{proof}

\subsection{Implementation}
The implementation is pretty straightforward using GHC's rewrite rules described in the work of Peyton Jones \cite{pbr}. A rewrite rule consist in a name and body. The body is an equality were free variables are bounded by a universal quantifier. When the left part of the equality is encountered in the code (where types matches), it is replaced by the right part. More details can be found on GHC's manual \cite{ghc:manual}.

A rewrite rule can also specify a phase constraint. GHC is performing 3 distinct phases where it tries to apply rewrite rules and it inlines (replace by its definition) functions. The so-called "phase control" is the \emph{only} mechanism that allows us to control when the rules are firing and when functions are inlined.

We want to have our rules compatible with those for lists already in place (the \verb|foldr/build| of Gill et al. \cite{Gill:1993:SCD:165180.165214}). They all target the \minline{bind} (defined in\verb|GHC.Base|) functions that is inlined in phase 1, so we have to do all our rewriting work before, in phase 2.

Fortunately, this is not very hard work. First of all, imitating again the work made for lists, we will add an
\begin{minted}{Haskell}
{-# INLINE [1] buildR #-}
\end{minted}
pragma, to ensure that \minline{buildR} is not inlined too quickly. Note that we force the inlining by using \verb|INLINE| and not only \verb|NOINLINE|, but this is a good idea since high-order functions like \minline{buildR} generally benefit from inlining.

Now, our rule will also target \minline{cata}, do we also add an \verb|INLINE| pragma to it.
\begin{minted}{Haskell}
{-# INLINE [0] cata #-}
\end{minted}

Now we can safely add the rule:
\begin{minted}{Haskell}
type Cata f = forall b. (f b -> b) -> b
{-# RULES
"cata/buildR" forall (f::t b -> b) (g::Cata t).
  cata f (buildR g) = g f
 #-}
\end{minted}
This rule is genuinely replacing all occurrences of\\ \minline{cata f (buildR g)} by \minline{g f} \emph{when types matches}, that is when the argument of \minline{buildR} are allowing fusion.

\subsection{Rewrite unneeded destructions}
\label{sec:bind-def}
All our work if focused on destructing structures. It is working great but can have unwanted implications. Consider the following code, implementing \minline{bind} for recursive trees:

\begin{minted}{Haskell}
bind :: Tree a -> (a -> Tree b) -> Tree b
bind t f = buildR
  (\u -> cata
    (\x ->
      case x of
        EmptyF -> u EmptyF
        LeafF x -> cata u (f x)
        NodeF a b -> u (NodeF a b))
    t)
\end{minted}

So given a \minline{u} (which without rewriting will be \minline{Fix}), you can indeed go trough the tree until you find a leaf, and replace this leaf by an application of a function producing a tree (here called \minline{f}). But why do we need the unneeded \minline{cata u} after? Because one can replace \minline{u} by an other function than \minline{Fix}, for example, something that calculate the sum of the leaves of the tree, thus we effectively need to "destruct" the graph constructed by \minline{f}. The solution is to use a second rewriting rule that get rid of the identity catamorphism. Due to the work of Breitner et al. \cite{Breitner:2014:SZC:2692915.2628141} we cannot target \minline{Fix} in our rewrite rule since it will be optimized away. Fortunately, this use case was planned (section 6.5 of the previously mentioned article) and we can target the function that replace \minline{Fix}: \minline{coerce}.

\begin{minted}{Haskell}
type Cata f = forall b. (f b -> b) -> b
{-# RULES
"cata/id" forall (g :: Fix f)
  cata coerce g = g
#-}
\end{minted}

As a side note, this little problem induce a possible limitation but easily solved: the need for optimizations. Rewrite rules mechanism are enabled either when required by the user or where optimization is enabled; fortunately, \verb|cabal| enable it by default. Either, a problem with a direct use of \minline{build} in \minline{bind} body will introduce an unwanted and unneeded traversal of the part of the graph as showed. This is solved (and this is the solution taken by the \verb|GHC.Base| module), it to define "normally" \minline{bind}, and add a rewrite rule to rewrite it in its "build" form.

Note also that, as Voigtländer \cite{Voigtlnder2008TypesFP} noticed for \minline{map}, \minline{bind} for \minline{Tree} is expressed both with \minline{buildR} and \minline{cata}, thus it can be fused with \emph{both} producers and consumers of trees.

\subsection{Examples}
\subsubsection{fold/build}
The \verb|foldr/build| rule is now just a particular case, using:
\begin{minted}{haskell}
map :: (a -> b) -> List a -> List b
map f xs = buildR $ \n ->
  let go x = case x of
    NilF -> n NilF
    ConsF a b -> n (ConsF (f a) b)
  in cata go xs
\end{minted}

\noindent The expression:
\begin{minted}{Haskell}
foldr c e (map f xs)
\end{minted}
(with \minline{foldr} defined as in section \ref{subsec:defi}) will first be rewritten (only by inlining) as:
\begin{minted}{haskell}
cata go' $ buildR $ \n ->
  let go x = case x of
    NilF -> n NilF
    ConsF a b -> n (ConsF (f a) b)
  in cata go xs
\end{minted}

\noindent With the \verb|cata/buildR| rule this will be rewritten to:
\begin{minted}{haskell}
let go x = case x of
  NilF -> go' NilF
  ConsF a b -> go' (ConsF (f a) b) in
cata go xs
\end{minted}
That is exactly what we wanted!

\subsubsection{Hylomorphism}
We can define \minline{ana} in term of \minline{buildR}:
\begin{minted}{Haskell}
ana :: Functor f => (b -> f b) -> b -> Fix f
ana f b = buildR $ \comb ->
  let c = comb . fmap c . f in c b
\end{minted}

\noindent Then fusion of catamorphism composed with an anamorphism is automatic !

\subsection{Compatibility with GHC's rules on lists}
Our work is compatible with the one already made in the \verb|GHC.Base| and \verb|GHC.List| modules (described in the work of Peyton Jones et al. \cite{playing-by-the-rules-rewriting-as-a-practical-optimisation-technique-in-ghc} ) in the sense that one can write a function that destruct a list (using \verb|foldr| or any "good consumer" of lists) to a fixed point structure \emph{that can be fused with list producers and structure consumers}.

Let us take an example with \minline{Tree}, by defining\\ \minline{treeFromList} that convert a list to a degenerate tree (where each node has a leaf):
\begin{minted}{Haskell}
treeFromList :: [Tree a] -> Tree a
treeFromList xs = buildR $ \g ->
  foldr
    (\x -> g . NodeF (cata g x))
    (g EmptyF)
    xs
\end{minted}
\noindent \minline{treeFromlist} is made with both \minline{foldr} and \minline{buildR}, so it can fuse with both \minline{cata} or \minline{build}.

The reverse is also possible, for example to convert a fixed-point list to a natural one:
\begin{minted}{Haskell}
flistToList :: List a -> [a]
flistToList xs = GHC.Base.build $ \c n ->
  let go xs = case xs of
    NilF -> n
    ConsF a b -> c a b
  in cata go xs
\end{minted}
\noindent \minline{treeFromlist} is made with both \minline{cata} and \minline{build}, so it can fuse with both \minline{buildR} or \minline{foldr}.

It is simple to see that the system composed by \verb|cata/buildR| and \verb|fold/build| rules is \emph{confluent}, that is, if there is an expression where both rules can fire, the order is not important (because both rules are just like a $\beta$-reduction with some manipulation of arguments).

\section{Going further with paramorphisms}
\label{sec:para}
%!TEX root = fusion.tex
\subsection{Paramorphisms}
\label{sec:para-list}

As far as we did, we worked only on catamorphisms. These are not really satisfying to define recursive functions that also uses the unconsumed part of the structure. A \emph{paramorphism} (introduced by Meertens \cite{Meertens1992}) is a direct response, as it allows the combination function to have a look on the unconsumed part of the structure, imitating the full power of recursion.

They can be defined for lists using \minline{paraL}:
\begin{minted}{Haskell}
paraL :: (a -> [a] -> b -> b) -> b -> [a] -> b
paraL _ z []     = z
paraL k z (x:xs) = k x xs (para k z xs)
\end{minted}

Again, we can also express paramorphisms with types defined in terms of \minline{Fix}, but using a little trick: The usual way (as in the \verb|recursion-schemes| package \cite{ekmett:eschems}) is with \minline{para}:
\begin{minted}{Haskell}
para :: Functor f =>
  (f (Fix f, a) -> a) -> Fix f -> a
para f = c
  where
    c (Fix x) = f (fmap (\y -> (y,p y)) x)
\end{minted}

\noindent What is happening here? As with \minline{cata}, when we are trying to destruct a recursive structure, we are doing the recursive step plus giving a copy of the original data in a tuple.
For example, we can encode a classic paramorphism, which is giving the list of proper suffixes of a list:
\begin{minted}{Haskell}
suff :: List a -> List (List a)
suff = para go
  where
    go NilF = cons nil nil
    go (ConsF x (xs,b)) = cons (cons x xs) b
\end{minted}

\subsection{Fusion for paramorphisms}

Now consider the less trivial paramorphism \minline{has2Leaves} defined in figure \ref{fig:has2Leaves} which searches in a binary tree if there is a node where the first leaf is in the left part, and the second in the right one.

Imagine you are composing it with a \minline{mapt} operation (as defined in section \ref{fig:has2Leaves}). Then
the function:
\begin{minted}{Haskell}
has2LeavesBind :: Eq b =>
  b -> b -> (a -> b) -> Tree a -> Bool
has2LeavesBind u v f = has2leaves u v . mapt f
\end{minted}
\noindent is equivalent to:
\begin{minted}{Haskell}
has2LeavesBind :: Eq b =>
  b -> b -> (a -> b) -> Tree a -> Bool
has2LeavesBind u v f t = Edge == para go t
  where
    go EmptyF    = Miss
    go (LeafF x) =
      if f x == u then Tail else Miss
    go (NodeF (_,a') (b,b')) =
      case a' of
        Miss -> b'
        Tail -> if cata go' b then Edge else Tail
        Edge -> Edge
    go' EmptyF      = False
    go' (LeafF x)   = f x == v
    go' (NodeF a b) = a || b
\end{minted}

\noindent Can we generalize the work made in section \ref{sec:rectypes} to make this work? The answer is yes (section \ref{sec:para-impl}), but let us first examine more deeply the process.

\begin{figure}
\begin{minted}{Haskell}
data Hit = Miss | Tail | Edge
  deriving (Eq, Ord, Show)

hasLeaf :: Eq a => a -> Tree a -> Bool
hasLeaf s = cata go
  where
    go EmptyF      = False
    go (LeafF x)   = x == s
    go (NodeF a b) = a || b

has2Leaves :: Eq a => a -> a -> Tree a -> Bool
has2Leaves u v = (==) Edge . para go
  where
    go EmptyF = Miss
    go (LeafF x) =
      if x == u then Tail else Miss
    go (NodeF (_,a') (b,b')) =
      case a' of
        Miss -> b'
        Tail -> if hasLeaf v b then Edge else Tail
        Edge -> Edge

mapt :: (a -> b) -> Tree a -> Tree b
mapt f t = bind t (\x -> fix (LeafF f x))
\end{minted}
\caption{Functions on trees}
\label{fig:has2Leaves}
\end{figure}

\subsection{Usability and validity}
As noticed by Domínguez and Pardo. \cite{paramorphismFusion} fusing paramorphisms with other operations is not always a good idea. Consider the previous example with proper suffixes of a list, but composed with a map:
\begin{minted}{Haskell}
suff (map f xs)
\end{minted}
\noindent If we try to fuse the two functions, we obtain:
\begin{minted}{Haskell}
let go x =
  case x of
    NilF -> nil
    ConsF x (xs,b) -> 
      cons (map f (cons x xs)) b
  in para go xs
\end{minted}
\noindent The \minline{map} operation happened for each suffix! The problem is that the combination function used \emph{both} the recursive call and the unconsumed part of the structure, which leads to a duplication of the original \minline{map} operation.

Sadly, this restriction is not easy to encode in types for \minline{Fix}: a valid function may inspect the recursive call on a part of the structure, then decide if it will use another recursive call or another unconsumed part of the structure; dependent types might be a solution, but this is not in our scope. The implementation below does not take into account this problem and allow fusion of any paramorphism.

\subsection{Implementation}
\label{sec:para-impl}

It is well-known that every catamorphism is a paramorphism where the combination function don't use the unconsumed structure. Indeed, you can define \minline{cata} as:
\begin{minted}{Haskell}
cata :: Functor f => (f a -> a) -> Fix f -> a
cata go = para (go . fmap snd)
{-# INLINE cata #-}
\end{minted}

\noindent It is then tempting to define a rule that will only target paramorphisms. However, paramorphism fusion is not as classy as the catamorphisms one, and having only one rule implies two main problems:
\begin{itemize}
\item We cannot anymore target \minline{cata coerce}, since \minline{cata} is just \minline{para}. We cannot either target\\ \minline{para (coerce . fmap snd)}, since \minline{fmap} is a class function; it is consequently difficult to avoid unneeded destruction, such as the one we can encounter when rewriting a \minline{bind} in its \minline{buildR} form.

\item It is not compatible with GHC's \minline{build}, as discussed in section \ref{sec:parabuild}.
\end{itemize}

We consequently choose to target a specific function that represent either a catamorphism or a paramorphism, called \minline{destruct}, defined with according rules in figure \ref{fig:para}. \minline{destruct} takes a combination function either for a catamorphism or for a paramorphism, and inline accordingly at phase 1.
This choice has obviously some consequences, each described below.

\begin{figure}
\begin{minted}{Haskell}
type Combi f a =
  Either (f a -> a) (f (Fix f, a) -> a)
  
type Para f = forall a. Combi f a -> a 

destruct :: Functor f => Combi f a -> Fix f -> a
destruct (Left c)  = cata c
destruct (Right c) = para c
{-# INLINE [1] destruct #-}

buildR :: Functor f => Para f -> Fix f
buildR g = g (Left fix)
{-# INLINE [1] buildR #-}

{-# RULES
"cata/buildR" [~1] forall f (g::Para f).
  cata f (buildR g) = g (Left f)

"para/buildR" [~1] forall f (g::Para f).
  para f (buildR g) = g (Right f)

"destruct/buildR" [~1] forall f (g::Para f).
  destruct f (buildR g) = g f

"cata/id"
  cata coerce = id
 #-}
\end{minted}
\caption{The rule system for paramorphism fusion}
\label{fig:para}
\end{figure}

\subsubsection{For the user, the example of bind on binary trees}
Let us take back our \minline{has2leaves}  but composed with a previous operation on the tree \emph{that preserved} the structure. It is not difficult to see that such an operation can be written using \minline{bind}, so let us focus ourselves on an expression like:

\begin{minted}{Haskell}
has2Leaves u v (bind t f)
\end{minted}

\noindent For fusion to happen, we need to have \minline{bind} on a re-writeable form for a paramorphism, that is to write it in term of \minline{buildR}. The interesting part comes with the fact that \minline{bind} appears also on its \minline{buildR} form, as you can see in figure \ref{fig:bindbuild}. It is necessary since we need to perform the \minline{bind} operation also on the unconsumed part of the tree. Since \minline{has2Leaves} also destruct the unconsumed part of the tree using \minline{hasLeaf} (a catamorphism), one can hope that fusion will also happen here, so we need this \minline{bind} to be also in a rewritable form, so we introduce \minline{bindR} which is just \minline{bind} in its (final) \minline{buildR} form.

This leads to a problem: to cover all cases, we need to "unroll" \minline{build} in its \minline{buildR} form indefinitely. Indeed, one can use another paramorphism on an unconsumed part of the structure. For us, these functions are maybe not of a big interest, and the rule system presented here will allow fusion only if the unconsumed part of the structure is destructed by a catamorphism (like in \minline{has2Leaves}).

\begin{figure}
\begin{minted}{Haskell}
bind :: Tree a -> (a -> Tree b) -> Tree b
bind t f = buildR $ \u -> destruct (
  case u of
    Left u -> Left $ \x -> 
      case x of
        EmptyF -> u EmptyF
        LeafF x -> cata u (f x)
        NodeF a b -> u $ NodeF a b)
    Right u -> Right $ \x ->
      case x of
        EmptyF -> u EmptyF
        LeafF x -> para u (f x)
        NodeF (a,a') (b,b') -> u $
          NodeF (bindR f a,a') (bindR f b, b'))
  t

-- This is just bind
bindR :: Tree a -> (a -> Tree b) -> Tree b
bindR t f = buildR $ \u -> destruct (
  case u of
    Left u -> Left $ \x -> 
      case x of
        EmptyF -> u EmptyF
        LeafF x -> cata u (f x)
        NodeF a b -> u $ NodeF a b)
    Right u -> Right $ \x ->
      case x of
        EmptyF -> u EmptyF
        LeafF x -> para u (f x)
        NodeF (a,a') (b,b') -> u $
          NodeF (cata go a,a') (cata go b, b'))
  t
  where
    go EmptyF = fix EmptyF
    go (LeafF a) = f a
    go (NodeF a b) = fix (NodeF a b)

\end{minted}
\caption{\minline{bind} in term of \minline{bindR} and \minline{destruct}}
\label{fig:bindbuild}
\end{figure}

\subsubsection{Many rules}
We now have three rules, and we need all of them. Indeed, \verb|"cata/buildR"| and \verb|"para/buildR"| are needed to correctly fuse the two recursion schemes. \verb|"destruct/buildR"| is a rule needed when composing functions. For example with an expression like:
\begin{minted}{Haskell}
bind (bind t f) g
\end{minted}
\noindent the two \minline{bind} will be inlined into their \minline{buildR} form, but since they have to deal with either a paramorphism or a catamorphism, they need to do a case analysis on the combination function. A possibility is to perform directly the case analysis, but this will duplicate the occurrences of the second \minline{bind} into the body of the first, and GHC is smart enough to create a variable to reduce code duplication, no knowing that it will be optimized after. So we can use \minline{destruct}, and performing the case analysis in its argument. Hence, without a too early inlining of it, the second \minline{bind} appears only once, and we can fuse it using \verb|"destruct/buildR"|.

\subsubsection{Compatibility with GHC's rules on lists}
\label{sec:parabuild}
To be able to fuse eligible paramorphisms with functions that are working on lists, one need to have access to paramorphisms \emph{for lists}. Moreover, these paramorphisms must be expressed with \minline{foldr}, since we want to preserve list-fusion with the \verb|"foldr/build"| rule. The solution is to give a copy of the structure at each step. Sadly, this definition is not what we need. Indeed, imagine that you want, for example, produce a fixed-point list from a classical list, but allowing fusion with \emph{both} list producers and \minline{List} consumers, you will have to do something like:
\begin{minted}{Haskell}
fromClassicalList :: [a] -> List a
fromClassicalList xs = 
  buildR $ \(Right u) -> snd $
    foldr 
      (\x (xs,y) -> 
        (cons x xs, u $ ConsF x $ (xs,y)))
      (nil,(u NilF))
    xs
\end{minted}
Excuse the non-exhaustive pattern matching, this is the only interesting case. So the obvious problem is that, even if \minline{u} is guaranteed to not use both the not-consumed structure and the recursive call, they need to be both available. The medication is here worse than the disease, and if we perform fusion with lists, we must do it with, for now, only catamorphisms.

% + TODO relire papier PeytonJones (playing by the rules)

\section{Related and further work}
\label{sec:related}
%!TEX root = fusion.tex

There are several improvements that can be made:
\begin{itemize}
\item As mentioned by Peyton Jones et al. \cite{pbr}, forcing the inlining of functions be able to rewrite them can produce a lot of code. A possible solution is to introduce a rule per function that rewrite them back if fusion did not happen.
\item \minline{seq} and $\bot$ does not interact well with our rules, studying their impact and how to recover semantic equivalence in this precise case would be interesting.
\item As explained, fusion of paramorphisms is not always a good idea. It would be great to have a convenient type criterion that makes the fusion interesting from a performance point of view.
\item We studied only two rules system, maybe others can be adapted to work on fixed-point structures.
\end{itemize}

\begin{acks}
TODO
\end{acks}


\bibliographystyle{ACM-Reference-Format}
\bibliography{publications}
\end{document}

