%!TEX root = fusion.tex

Deforestation is a process first introduced by Wadler \cite{WADLER1990231} and realized in Haskell by Gill et al. \cite{Gill:1993:SCD:165180.165214}. It is the idea that one can eliminate a structure produced by "good function" (like an anamorphism) and destructed right away by a catamorphism.

The first example of this concept is on lists with the expression:
\begin{minted}{Haskell}
foldr c e (map f xs)
\end{minted}
\minline{map} will destruct the original list and produce a new one, which will be destructed after with \minline{foldr}. The key thing is that we know how \minline{map} will construct the new list. Obviously, the above expression is equivalent to a more efficient one:
\begin{minted}{Haskell}
foldr (c . f) e xs
\end{minted}
The above expression will go through the list only once.
The rewriting operation we made in this example is again not at all dependent of the list structure: we just used the fact that we know how is made the list produced by \minline{map}. This concept can be simply generalized to recursive structure, as we will show in the next section.

\subsection{The idea}
Deforestation can be implemented on recursive types imitating the work of Gill et al. \cite{Gill:1993:SCD:165180.165214}. Applying \minline{cata go} to a recursive structure has just the effect to replace every occurrence of \minline{Fix} by \minline{go} in the structure. Consequently, the idea is to abstract structure-producing functions with respect to \minline{Fix}. This is done using \minline{buildR}.

\begin{minted}{Haskell}
type Cata f = forall b. (f b -> b) -> b

buildR :: Cata f -> Fix f
buildR g = g Fix
\end{minted}

\noindent Intuitively, the argument of \minline{buildR} is a function that can, for all type \minline{b}, and given something to "perform recursion" (that is, produce a \minline{b} from a \minline{(f b)}) can produce something of type \minline{b}. For example, you can use \minline{buildR} to define \minline{ana}:
\begin{minted}{Haskell}
ana :: Functor f => (b -> f b) -> b -> Fix f
ana f b = buildR
  (\u -> let c =  u . fmap c . f in c b)
\end{minted}

\subsubsection{Validity}
Now, we have the lovely property that:
$$cata\ go\ (buildR\ g)\ =\ g\ go$$

This is exactly what we want: the right-side of the equation \emph{does not} build an intermediate structure. The validity of the above rule is guaranteed by the type (more precisely the rank-2 one).

\begin{theorem}
Using, for a fixed parametric type $F$, a valid instance of $Functor$, and a type $A$:
\begin{itemize}
	\item $g :: \forall\ B.\ (F\ B \to B) \to B$
	\item $go :: F\ A \to A $
\end{itemize}
$$cata\ go\ (buildR\ g)\ =\ g\ go$$
\end{theorem}
\begin{proof}
This is a consequence of free theorems of Wadler \cite{Wadler:1989:TF:99370.99404}, as for the \verb|foldr/build| rule \cite{Gill:1993:SCD:165180.165214}. The free theorem associated with $g$'s type is that, for all types $C$ and $D$, $f\ ::\ C \to D$, $p\ ::\ F\ C \to C$ and $q\ ::\ F\ D \to D$, then we have the following implication:
\begin{align*}
[\ \forall& x::C.\ f\ (p\ x)\ =\ q\ (fmap\ f\ x)\ ]\\
&\implies f\ (g\ p)\ =\ g\ q
\end{align*}
Now we can pose:
\begin{itemize}
	\item $C\ =\ Fix\ F$, $D\ =\ A$
	\item $f\ =\ cata\ go$
	\item $p\ =\ fix$
	\item $q\ =\ go$
\end{itemize}
then we have
\begin{align*}
&[\ \forall x::F\ (Fix\ F).\\
&cata\ go\ (fix\ x)\ =\ go\ (fmap\ (cata\ go)\ x)\ ]\\
&\implies cata\ go\ (g\ fix)\ =\ g\ go
\end{align*}
Given the definition of $cata$, the premise trivially holds. So we have the conclusion:
$$cata\ go\ (g\ fix)\ =\ g\ go$$
Given the definition of $buildR$, this is exactly what we required.
\end{proof}

\subsubsection{A note on $\bot$ and \minline{seq}}
TODO

\subsection{Implementation}
The implementation is pretty straightforward using GHC's rewrite rules described in the great work of Peyton Jones et al. \cite{pbr}. A rewrite rule consist in a name and body. The body is an equality were free variables are bounded by a universal quantifier. When the left part of the equality is encountered in the code (and types match), it is replaced by the right part. More details can be found on GHC's manual \cite{ghc:manual}.

A rewrite rule can also specify a phase constraint. GHC is performing 3 distinct phases where it tries to apply rewrite rules and it inlines (replace by its definition) functions. The so-called "phase control" is the \emph{only} mechanism that allows us to ensure when the rules are firing and when functions are inlined.

We want to have our rules compatible with those for lists already in place (the \verb|foldr/build| of Gill et al. \cite{Gill:1993:SCD:165180.165214}). They all target the \minline{bind} (defined in\verb|GHC.Base|) functions that is inlined in phase 1, so we have to do all our rewriting work before, in phase 2. Fortunately, this is not very hard work. First of all, imitating again the work made for lists, we will add a
\begin{minted}{Haskell}
{-# INLINE [1] buildR #-}
\end{minted}
pragma, to ensure that \minline{buildR} is not inlined too quickly. Note that we force the inlining by using \verb|INLINE| and not only \verb|NOINLINE|, but this is a good idea since high-order functions like \minline{buildR} generally benefit from inlining.

Now, our rule will also target \minline{cata}, we consequently also need to add a \verb|INLINE| pragma to it.
\begin{minted}{Haskell}
{-# INLINE [0] cata #-}
\end{minted}

\noindent Now we can safely add the rule:
\begin{minted}{Haskell}
type Cata f = forall b. (f b -> b) -> b
{-# RULES
"cata/buildR" [~1]
  forall (f::t b -> b) (g::Cata t).
    cata f (buildR g) = g f
 #-}
\end{minted}
This rule is genuinely replacing all occurrences of\\ \minline{cata f (buildR g)} by \minline{g f} \emph{when types matches}, that is when the argument of \minline{buildR} are allowing fusion. The \minline{[~1]} parameter will disable the rule in the final phase, so we can rewrite un-merged functions back.

\subsection{Rewrite unneeded destructions}
\label{sec:bind-def}
All our work if focused on destructing structures. It is working great but can have unwanted implications. Consider the following code, implementing \minline{bind} for recursive trees:

\begin{minted}{Haskell}
bind :: Tree a -> (a -> Tree b) -> Tree b
bind t f = buildR
  (\u -> cata
    (\x ->
      case x of
        EmptyF -> u EmptyF
        LeafF x -> cata u (f x)
        NodeF a b -> u (NodeF a b))
    t)
\end{minted}

\noindent Given a \minline{u} (which without rewriting will be \minline{Fix}), you can indeed go trough the tree until you find a leaf, and replace this leaf by an application of a function producing a tree (here called \minline{f}). But why do we need the unneeded \minline{cata u} after? Because one can replace \minline{u} by another function than \minline{Fix}, for example, something that calculate the sum of the leaves of the tree, thus we effectively need to "destruct" the graph constructed by \minline{f}. The solution is to use a second rewriting rule that get rid of the identity catamorphism. Due to the work of Breitner et al. \cite{Breitner:2014:SZC:2692915.2628141} we cannot target \minline{Fix} in our rewrite rule since it will be optimized away. Fortunately, this use case was planned (section 6.5 of the previously mentioned article) and we can target the function that replace \minline{Fix}: \minline{coerce}.

\begin{minted}{Haskell}
{-# RULES
"cata/id" forall (g :: Fix f)
  cata coerce g = g
#-}
\end{minted}

\noindent As a side note, this little problem induce a possible limitation but easily solved: the real need for optimizations. Rewrite rules mechanism are enabled either when required by the user or where optimization is enabled; fortunately, \verb|cabal| enable it by default. If rewrite are not enabled, the direct use of \minline{build} in \minline{bind} body will introduce an unwanted and unneeded traversal of the part of the graph as showed. This is solved (and this is the solution taken by the \verb|GHC.Base| module), it to define "normally" \minline{bind}, and add a rewrite rule to rewrite it in its \minline{build} form.

Note also that, as Voigtländer \cite{Voigtlnder2008TypesFP} noticed for \minline{map}, \minline{bind} for \minline{Tree} is expressed both with \minline{buildR} and \minline{cata}, thus it can be fused with \emph{both} producers and consumers of trees.

\subsection{Examples}
\subsubsection{foldr/build}
The \verb|foldr/build| rule is now just a particular case, using:
\begin{minted}{haskell}
map :: (a -> b) -> List a -> List b
map f xs = buildR $ \u ->
  let go x = case x of
    NilF -> u NilF
    ConsF a b -> u (ConsF (f a) b)
  in cata go xs
\end{minted}

\noindent The expression:
\begin{minted}{Haskell}
foldr c e (map f xs)
\end{minted}
(with \minline{foldr} defined as in section \ref{subsec:defi}) will first be rewritten (only by inlining) as:
\begin{minted}{haskell}
cata go' $ buildR $ \u ->
  let go x = case x of
    NilF -> u NilF
    ConsF a b -> u (ConsF (f a) b)
  in cata go xs
\end{minted}

\noindent With the \verb|cata/buildR| rule this will be rewritten to:
\begin{minted}{haskell}
let go x = case x of
  NilF -> go' NilF
  ConsF a b -> go' (ConsF (f a) b) in
cata go xs
\end{minted}
That is exactly what we wanted!

\subsubsection{Automatic hylomorphisms}
We can define \minline{ana} in term of \minline{buildR}:
\begin{minted}{Haskell}
ana :: Functor f => (b -> f b) -> b -> Fix f
ana f b = buildR $ \u ->
  let c = u . fmap c . f in c b
\end{minted}

\noindent Then fusion of catamorphism composed with an anamorphism is automatic!

\subsection{Compatibility with GHC's rules on lists}
Our work is compatible with the one already made in the \verb|GHC.Base| and \verb|GHC.List| modules (described in the work of Peyton Jones et al. \cite{pbr} ) in the sense that one can write a function that destruct a list (using \verb|foldr| or any "good consumer" of lists) to a fixed point structure \emph{that can be fused with list producers and structure consumers}.

Let us take an example with \minline{Tree}, by defining\\ \minline{treeFromList} that convert a list to a degenerate tree (where each node has a leaf):
\begin{minted}{Haskell}
treeFromList :: [Tree a] -> Tree a
treeFromList xs = buildR $ \u ->
  foldr
    (\x -> u . NodeF (cata u x))
    (g EmptyF)
    xs
\end{minted}
\noindent \minline{treeFromlist} is made with both \minline{foldr} and \minline{buildR}, so it can fuse with both \minline{cata} or \minline{build}.

The reverse is also possible, for example to convert a fixed-point list to a natural one:
\begin{minted}{Haskell}
flistToList :: List a -> [a]
flistToList xs = GHC.Base.build $ \c n ->
  let go xs = case xs of
    NilF -> n
    ConsF a b -> c a b
  in cata go xs
\end{minted}
\noindent \minline{treeFromlist} is made with both \minline{cata} and \minline{build}, so it can fuse with both \minline{buildR} or \minline{foldr}.

It is simple to see that the system composed by \verb|cata/buildR| and \verb|foldr/build| rules is \emph{confluent}, that is, if there is an expression where both rules can fire, the order is not important (because both rules are just like a $\beta$-reduction with some manipulation of arguments).