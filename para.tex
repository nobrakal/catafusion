%!TEX root = fusion.tex
\subsection{Paramorphisms}
\label{sec:para-list}

For now, we worked only on catamorphisms. They are not really satisfying to define recursive functions that also use a part of the structure without destructing it recursively first. A \emph{paramorphism} (introduced by Meertens \cite{Meertens1992}) is a direct response to this problem, as it allows the combination function to have a look on the unconsumed part of the structure, imitating the full power of recursion.

They can be defined for lists using \hs{paraL}:
\begin{minted}{Haskell}
paraL :: (a -> [a] -> b -> b) -> b -> [a] -> b
paraL _ z []     = z
paraL k z (x:xs) = k x xs (para k z xs)
\end{minted}

\noindent Again, we can also express paramorphisms with types defined in terms of \hs{Fix}, but using a little trick. The usual way (as in the \verb|recursion-schemes| package \cite{ekmett:eschems}) is with \hs{para}:
\begin{minted}{Haskell}
para :: Functor f =>
  (f (Fix f, a) -> a) -> Fix f -> a
para f = c
  where
    c (Fix x) = f (fmap (\y -> (y,p y)) x)
\end{minted}

\noindent What is happening here? As with \hs{cata}, when we are trying to destruct a recursive structure, we are first doing the recursive step (using the \hs{Functor} instance), but we are giving in addition a copy of the original data in a tuple, offering them both to the combination function.
For example, we can encode a classic paramorphism, which is giving the list of suffixes of a list:
\begin{minted}{Haskell}
suff :: List a -> List (List a)
suff = para go
  where
    go NilF = cons nil nil
    go (ConsF x (xs,b)) = cons (cons x xs) b
\end{minted}

\subsection{Fusion for paramorphisms}

Now consider the less trivial paramorphism \hs{has2Leaves} defined in Figure \ref{fig:has2Leaves} which is searching in a binary tree if there is a node where the first leaf is in the left part, and the second in the right one. This kind of paramorphisms arise naturally when working with more complex recursive structures, like algebraic graphs of Mokhov \cite{Mokhov:2017:AGC:3156695.3122956}.

Now we are trying to compose it with a \hs{mapt} (as defined in Section \ref{fig:has2Leaves}). Then the function:
\begin{minted}{Haskell}
has2LeavesMapt :: Eq b =>
  b -> b -> (a -> b) -> Tree a -> Bool
has2LeavesMapt u v f = has2leaves u v . mapt f
\end{minted}
\noindent is equivalent to:
\begin{minted}{Haskell}
has2LeavesMapt :: Eq b =>
  b -> b -> (a -> b) -> Tree a -> Bool
has2LeavesMapt u v f t = Edge == para go t
  where
    go EmptyF    = Miss
    go (LeafF x) =
      if f x == u then Tail else Miss
    go (NodeF (_,a') (b,b')) =
      case a' of
        Miss -> b'
        Tail -> if cata go' b then Edge else Tail
        Edge -> Edge
    go' EmptyF      = False
    go' (LeafF x)   = f x == v
    go' (NodeF a b) = a || b
\end{minted}

\noindent Can we generalize what we made in Section \ref{sec:rectypes} to make this work? The answer is yes (Section \ref{sec:para-impl}), but let us first examine more deeply the process.

\begin{figure}
\begin{minted}{Haskell}
data Hit = Miss | Tail | Edge
  deriving (Eq, Ord, Show)

hasLeaf :: Eq a => a -> Tree a -> Bool
hasLeaf s = cata go
  where
    go EmptyF      = False
    go (LeafF x)   = x == s
    go (NodeF a b) = a || b

has2Leaves :: Eq a => a -> a -> Tree a -> Bool
has2Leaves u v = (==) Edge . para go
  where
    go EmptyF = Miss
    go (LeafF x) =
      if x == u then Tail else Miss
    go (NodeF (_,a') (b,b')) =
      case a' of
        Miss -> b'
        Tail ->
          if hasLeaf v b then Edge else Tail
        Edge -> Edge

mapt :: (a -> b) -> Tree a -> Tree b
mapt f t = bind t (\x -> Fix (LeafF f x))
\end{minted}
\caption{Functions on trees}
\label{fig:has2Leaves}
\end{figure}

\subsection{Usability and validity}
As noticed by Domínguez and Pardo \cite{paramorphismFusion} fusing paramorphisms with other operations is not always a good idea. Consider the previous example with proper suffixes of a list, but composed with a map:
\begin{minted}{Haskell}
suff (map f xs)
\end{minted}
\noindent If we try to fuse the two functions, we obtain:
\begin{minted}{Haskell}
let go x =
  case x of
    NilF -> nil
    ConsF x (xs,b) ->
      cons (map f (cons x xs)) b
  in para go xs
\end{minted}
\noindent The \hs{map} operation is applied to each suffix! The problem is that the combination function used \emph{both} the recursive call and the unconsumed part of the structure, which leads to a duplication of the original \hs{map} operation.

Sadly, this restriction is not easy to encode in types for \hs{Fix}: a valid function may inspect the recursive call on a part of the structure, then decide if it will use another recursive call or another unconsumed part of the structure. The implementation below does not take into account this problem and allow fusion of any paramorphism.

\subsection{Implementation}
\label{sec:para-impl}

It is well-known that every catamorphism is a paramorphism where the combination function don't use the unconsumed structure. Indeed, one can define \hs{cata} as:
\begin{minted}{Haskell}
cata :: Functor f => (f a -> a) -> Fix f -> a
cata go = para (go . fmap snd)
{-# INLINE cata #-}
\end{minted}

\noindent It is then tempting to define a rule that will only target paramorphisms. However, paramorphism fusion is not as classy as the catamorphisms one, and having only one rule implies two main problems:
\begin{itemize}
\item We cannot anymore target \hs{cata coerce}, since \hs{cata} is just \hs{para}. We cannot either target\\ \hs{para (coerce . fmap snd)}, since \hs{fmap} is a class function; it is consequently difficult to avoid unneeded destructions, such as the one we can encounter when rewriting a \hs{bind} in its \hs{buildR} form.

\item It is not compatible with GHC's \hs{build}, as discussed in Section \ref{sec:parabuild}.
\end{itemize}

We consequently choose to target a specific function that represent either a catamorphism or a paramorphism, called \hs{destruct}, defined with according rules in Figure \ref{fig:para}. \hs{destruct} takes a combination function either for a catamorphism or for a paramorphism, and inline accordingly at phase 1.
This choice has obviously some consequences, each described below.

\begin{figure}
\begin{minted}{Haskell}
type Combi f a =
  Either (f a -> a) (f (Fix f, a) -> a)

type Para f = forall a. Combi f a -> a

destruct :: Functor f => Combi f a -> Fix f -> a
destruct (Left c)  = cata c
destruct (Right c) = para c
{-# INLINE [1] destruct #-}

buildR :: Functor f => Para f -> Fix f
buildR g = g (Left fix)
{-# INLINE [1] buildR #-}

{-# RULES
"cata/buildR" [~1] forall f (g::Para f).
  cata f (buildR g) = g (Left f)

"para/buildR" [~1] forall f (g::Para f).
  para f (buildR g) = g (Right f)

"destruct/buildR" [~1] forall f (g::Para f).
  destruct f (buildR g) = g f

"cata/id"
  cata coerce = id
 #-}
\end{minted}
\caption{The rule system for paramorphism fusion}
\label{fig:para}
\end{figure}

\subsubsection{For the user, the example of bind on binary trees}
Let us take back our \hs{has2leaves}  but composed with a previous operation on the tree \emph{that preserved} the structure. It is not difficult to see that such an operation can be written using \hs{bind}, so let us focus ourselves on an expression like:

\begin{minted}{Haskell}
has2Leaves u v (bind t f)
\end{minted}

\noindent For fusion to happen, we need to have \hs{bind} on a re-writeable form for a paramorphism, that is to write it in term of \hs{buildR}. The interesting part comes with the fact that \hs{bind} appears also on its \hs{buildR} form, as we can see in Figure \ref{fig:bindbuild}. It is necessary since we need to perform the \hs{bind} operation also on the unconsumed part of the tree. Since \hs{has2Leaves} also destructs the unconsumed part of the tree using \hs{hasLeaf} (a catamorphism), one can hope that fusion will also happen here, so we need this \hs{bind} to be also in a rewritable form, hence we introduce \hs{bindR} which is just \hs{bind} in its (final) \hs{buildR} form. This allows the \hs{has2leavesMapt} optimization described in the previous section to happen automatically.

This leads to a problem: to cover all cases, we need to "unroll" \hs{build} in its \hs{buildR} form indefinitely. Indeed, one can use another paramorphism on an unconsumed part of the structure, and hopes that fusion happen also here with the \hs{bind} made on the unconsumed part. For us, these functions are maybe not of a big interest, and the rule system presented here will allow fusion only if the unconsumed part of the structure is destructed by a catamorphism (like in \hs{has2Leaves}).

\begin{figure}
\begin{minted}{Haskell}
bind :: Tree a -> (a -> Tree b) -> Tree b
bind t f = buildR $ \u -> destruct (
  case u of
    Left u -> Left $ \x ->
      case x of
        EmptyF -> u EmptyF
        LeafF y -> cata u (f y)
        NodeF a b -> u $ NodeF a b)
    Right u -> Right $ \x ->
      case x of
        EmptyF -> u EmptyF
        LeafF y -> para u (f y)
        NodeF (a,a') (b,b') -> u $
          NodeF (bindR f a,a') (bindR f b, b'))
  t

-- This is just bind
bindR :: Tree a -> (a -> Tree b) -> Tree b
bindR t f = buildR $ \u -> destruct (
  case u of
    Left u -> Left $ \x ->
      case x of
        EmptyF -> u EmptyF
        LeafF y -> cata u (f y)
        NodeF a b -> u $ NodeF a b)
    Right u -> Right $ \x ->
      case x of
        EmptyF -> u EmptyF
        LeafF y -> para u (f y)
        NodeF (a,a') (b,b') -> u $
          NodeF (cata go a,a') (cata go b, b'))
  t
  where
    go EmptyF = fix EmptyF
    go (LeafF a) = f a
    go (NodeF a b) = fix (NodeF a b)

\end{minted}
\caption{\hs{bind} in term of \hs{bindR} and \hs{destruct}}
\label{fig:bindbuild}
\end{figure}

\subsubsection{Many rules}
We now have three rules, and we need all of them. Indeed, \verb|"cata/buildR"| and \verb|"para/buildR"| are needed to correctly fuse the two recursion schemes. \verb|"destruct/buildR"| is a rule needed when composing functions that transform the original structure. For example on binary trees, with an expression like:
\begin{minted}{Haskell}
bind (bind t f) g
\end{minted}
\noindent the two \hs{bind} will be inlined into their \hs{buildR} form, but since they have to deal with either a paramorphism or a catamorphism, they need to do a case analysis on the combination function. A possibility is to perform directly the case analysis, but this will duplicate the occurrences of the outer \hs{bind} into the body of the first one, and GHC is smart enough to create a variable to reduce code duplication, which prevents the \verb|cata/buildR| rule to fire, not knowing that it will be optimized after. So we can use \hs{destruct}, and perform the case analysis in its first argument. Hence, without a too early inlining of \hs{buildR}, the second \hs{bind} appears only once (in the second argument of \hs{destruct}), and we can fuse it using the \verb|"destruct/buildR"| rule.

\subsubsection{Compatibility with GHC's rules on lists}
\label{sec:parabuild}
To be able to fuse eligible paramorphisms with functions that are working on lists, one need to have access to paramorphisms \emph{for lists}. Moreover, these paramorphisms must be expressed with \hs{foldr}, since we want to preserve list-fusion with the \verb|"foldr/build"| rule. The classical solution is to give a copy of the list in the accumulating parameter of \hs{foldr}, so it is available for the "upper" combination function. Sadly, this definition is not what we need; indeed, imagine that we want to produce a fixed-point list from a classical list, but allowing fusion with \emph{both} Prelude list producers and \hs{List} consumers, we will have to do something like (excuse the non-exhaustive pattern matching, this is the only interesting case):
\begin{minted}{Haskell}
fromClassicalList :: [a] -> List a
fromClassicalList xs =
  buildR $ \(Right u) -> snd $
    foldr
      (\x (xs,y) ->
        (cons x xs, u $ ConsF x (xs,y)))
      (nil,(u NilF))
    xs
\end{minted}
\noindent So the obvious problem is that, even if \hs{u} is guaranteed to not use both the unconsumed structure and the recursive call, they need to be both available. The medication is here clearly worse than the disease: we are rebuilding the original list even if we don't use it. As a consequence, if we perform fusion with lists, we must do it carefully, and only with catamorphisms.

\subsection{Conclusion}
This rather more complicated rule system preserves the work made in Section \ref{sec:rectypes}, namely that catamorphisms can be fused with "good producers", and extends it to paramorphisms fusion with (almost) the same "good producers". However, all things have a price, and a major drawback is that the implementation and the definition of new functions is more difficult than with catamorphisms fusion alone.