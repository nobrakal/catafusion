\documentclass[format=sigplan, review=true, anonymous=true]{acmart}

\usepackage[utf8]{inputenc}
\usepackage[english]{babel}
\usepackage{amsmath}
\usepackage{amssymb}

\usepackage{minted}
\newcommand{\minline}[1]{\mintinline{Haskell}{#1}}

%
% defining the \BibTeX command - from Oren Patashnik's original BibTeX documentation.
\def\BibTeX{{\rm B\kern-.05em{\sc i\kern-.025em b}\kern-.08emT\kern-.1667em\lower.7ex\hbox{E}\kern-.125emX}}

% Rights management information.
% This information is sent to you when you complete the rights form.
% These commands have SAMPLE values in them; it is your responsibility as an author to replace
% the commands and values with those provided to you when you complete the rights form.
%
% These commands are for a PROCEEDINGS abstract or paper.
\copyrightyear{2019}
\acmYear{2019}
%\setcopyright{acmlicensed}
\acmConference[Haskell'19]{24th ACM SIGPLAN International Haskell Symposium}{September 18-23, 2019}{Berlin, Germany}

%\acmPrice{15.00}
%\acmDOI{10.1145/1122445.1122456}
%\acmISBN{978-1-4503-9999-9/18/06}

\begin{document}
\title{Deforestation for fixed-point structures}
\author{Alexandre Moine}
\email{alexandre@moine.me}

\affiliation{%
	\institution{Université Paris Diderot}
	\streetaddress{5 Rue Thomas Mann}
	\city{Paris}
	\country{France}
	\postcode{75013}
}

\begin{abstract}
Recursive structures are ubiquitous in Haskell, constructed and destructed continuously. It is known that such manipulations can sometimes be optimized to avoid the construction of intermediate structures with a process called deforestation. In this article, we propose a minimalistic automatic technique to perform it on fixed-point structures using the GHC rewrite rules system.
\end{abstract}

%
% The code below is generated by the tool at http://dl.acm.org/ccs.cfm.
% Please copy and paste the code instead of the example below.
% FROM https://dl.acm.org/ccs/ccs.cfm?id=10011012&lid=0.10011007.10011006.10011008.10011009.10011012

\begin{CCSXML}
	<ccs2012>
	<concept>
	<concept_id>10011007.10010940.10010992.10010993.10010994</concept_id>
	<concept_desc>Software and its engineering~Functionality</concept_desc>
	<concept_significance>300</concept_significance>
	</concept>
	<concept>
	<concept_id>10011007.10011006.10011008.10011009.10011012</concept_id>
	<concept_desc>Software and its engineering~Functional languages</concept_desc>
	<concept_significance>300</concept_significance>
	</concept>
	</ccs2012>
\end{CCSXML}

\ccsdesc[300]{Software and its engineering~Functionality}
\ccsdesc[300]{Software and its engineering~Functional languages}


%
% Keywords. The author(s) should pick words that accurately describe the work being
% presented. Separate the keywords with commas.

\keywords{Haskell, GHC, deforestation}

%
% This command processes the author and affiliation and title information and builds
% the first part of the formatted document.
\maketitle

\section{Introduction}

A Haskell program can be seen as the composition of many other small programs. It is often the right way to see software development but can lead to optimization problems. For example the expression:
\begin{minted}{Haskell}
foldr (+) 0 (map (* 2) xs)
\end{minted}
is equivalent to
\begin{minted}{Haskell}
foldr ((+) . (*) 2) 0 xs
\end{minted}

\noindent While the first one is more comprehensible, it builds an intermediate list that is not needed. The second version is optimized and will only go through the original list once.
This optimization, called \emph{deforestation} (introduced by Wadler \cite{WADLER1990231}) can be made at compile-time and thus allowing the user to write a readable code while the compiler optimize it when it can; this is known for a long time, well explained by Gill et al. \cite{Gill:1993:SCD:165180.165214} and implemented for lists and rose trees by Peyton Jones et al. \cite{pbr} for GHC.

Destruct a recursive structure (like \minline{foldr} for lists) is a very common operation and can be generally defined for fixed-point types (section \ref{sec:recschemes}), that is, for classical recursive types. Then, a legitimate question is to ask if we can perform deforestation, namely fusing a destruction with a construction of the original structure. Gill et al. \cite{Gill:1993:SCD:165180.165214} postulated that this was possible: we show that this is indeed achievable for fixed-point types (section \ref{sec:rectypes}). We also propose a minimalistic implementation that makes deforestation available to \emph{any} types defined using our fixed-point constructor. We prove in part the validity of technique and give specific examples.
Finally, we show that the approach taken can be extended to more complex fusion rules (section \ref{sec:para}).

\section{Recursion schemes}
\label{sec:recschemes}
%!TEX root = fusion.tex

Recursion schemes are almost everywhere in functional programming. They generalize two very common patterns: constructing a structure and destructing it recursively. They were popularized by Meijer et al. \cite{4cec4a43c86444479dc0003182424795}, which are giving them a categorical meaning. In this section, we review them in the particular case of fixed-point structures.

\subsection{Fixed-point structures}
\label{subsec:defi}
First of all, a classical recursive structure can be defined as a \emph{fixed-point}, using the \hs{Fix} data-type:
\begin{minted}{Haskell}
newtype Fix f = Fix (f (Fix f))
\end{minted}
\hs{Fix} directly encodes recursion in types: you are giving to a parametric type itself, again and again. Note that this data-type is defined with \hs{newtype} (and not simply with a \hs{type} synonym) since it calls itself. Fortunately, the use of a \hs{newtype} is optimized away by the compiler, using the great work of Breitner et al. \cite{Breitner:2014:SZC:2692915.2628141}, meaning that using \hs{Fix} in our code is at zero-cost.

For example, one can define, as in Figure \ref{fig:listtree},
\hs{(List a)}, isomorphic to \hs{[a]} and \hs{(Tree a)}, isomorphic to the classic recursive structure for binary trees:
\begin{minted}{Haskell}
data TreeB a =
    Empty
  | Leaf a
  | Node (Tree a) (Tree a)
\end{minted}

\begin{figure}
\begin{minted}{Haskell}
data ListF a b =
    NilF
  | ConsF a b

instance Functor (ListF a) where
  fmap _ NilF        = Nil
  fmap f (ConsF a b) = Cons a (f b)

type List a = Fix (List a)

nil :: List a
nil = Fix NilF

cons :: a -> List a -> List a
cons x xs = Fix (ConsF x xs)

data TreeF a b =
    EmptyF
  | LeafF a
  | NodeF b b

instance Functor (TreeF a) where
  fmap _ EmptyF      = EmptyF
  fmap _ (LeafF x)   = LeafF x
  fmap f (NodeF a b) = NodeF (f a) (f b)

type Tree a = Fix (TreeF a)
\end{minted}
\caption{Two fixed-point types: lists and binary trees}
\label{fig:listtree}
\end{figure}

\subsection{Catamorphisms}
A \emph{catamorphism} generalizes a fold on a recursive structure. For example, catamorphisms on lists can be expressed with the so-called \hs{foldr}.
\begin{minted}{Haskell}
foldr :: (a -> b -> b) -> b -> [a] -> b
foldr _ z []     = z
foldr k z (y:ys) = k y (foldr k z ys)
\end{minted}
Providing a definition for the empty list and another for the \verb|cons| operation allows us to go through a list and destruct it recursively. Note that we didn't use the concept of list but just its constructors; consequently this process applies to any other recursive type. The definition of a type in term of \hs{Fix} allows a straightforward implementation of \hs{cata}, a high-order function for catamorphisms.

\begin{minted}{Haskell}
cata :: Functor f => (f a -> a) -> Fix f -> a
cata f (Fix x) = f (fmap (cata f) x)
\end{minted}
\noindent How \hs{cata} is working? As any fold on any structure! Given a recursive structure, we will firstly destruct any substructure, and then applies a function to combine the recursive results. The "firstly destruct any substructure" step is performed by using a \hs{Functor} instance: it is the only way to get "inside" the structure, since we don't know how it is made. The combination function can also feel a bit strange but its type says it all: if we have a recursive structure where the substructures were replaced by a result, we must be able to combine them to produce a result.

For example, with \hs{(List a)} from the above example, one can write \hs{foldr}:
\begin{minted}{Haskell}
foldr :: (a -> b -> b) -> b -> List a -> b
foldr c e = cata go
  where
    go NilF = e
    go (ConsF a b) = c a b
\end{minted}

\noindent A better implementation of \hs{cata} due to E. Kmett can be found his \verb|recusrion-schemes| package \cite{ekmett:eschems}:

\begin{minted}{Haskell}
cata :: Functor f => (f b -> b) -> Fix f -> b
cata f = c where c (Fix x) = f (fmap c x)
\end{minted}

\noindent
where the recursive step was specialized to the \hs{f} argument.

\subsection{Anamorphisms}
An \emph{anamorphism} generalizes the notion of unfolding a recursive structure, that is, producing a structure from a seed value.
For example, anamorphisms on lists can be expressed with the so-called \hs{unfoldr}.
\begin{minted}{Haskell}
unfoldr :: (b -> Maybe (a, b)) -> b -> [a]
unfoldr f b =
  case f b of
    Just (a, b') -> cons a (unfoldr f b')
    Nothing      -> nil
\end{minted}

\noindent \hs{unfoldr} takes a function that, from a starting seed, produces maybe an element and another seed, or nothing. It then iterates the process until the function does not produce any more seed. As the clever reader already guessed, we can provide anamorphisms for any fixed-point structure, using \hs{ana} (as in the \verb|recursion-schemes| package \cite{ekmett:eschems}):
\begin{minted}{Haskell}
ana :: Functor f => (b -> f b) -> b -> Fix f
ana f b = c where c = Fix . fmap c . f
\end{minted}
The first argument is a function that, from a seed, produces a base structure filled with seeds. Then one can iterate the process to obtain a full structure. For example, we can write \hs{unfoldr} in term of \hs{ana} for our \hs{List} type:
\begin{minted}{Haskell}
unfoldr :: (b -> Maybe (a, b)) -> b -> List a
unfoldr f = ana go . f
  where
    go Nothing      = NilF
    go (Just (a,b)) = ConsF a (f b)
\end{minted}

\subsection{Fusion of recursive schemes - hylomorphisms}
A first nice result on these recursive schemes is that the composition of an anamorphism with a catamorphism can be optimized to remove the need of building the intermediate structure: we then obtain a \emph{hylomorphism} (as proved by Meijer et al. \cite{4cec4a43c86444479dc0003182424795}).
Hylomorphisms can be expressed using \hs{hylo} (as in the \verb|recursion-schemes| package \cite{ekmett:eschems}):
\begin{minted}{Haskell}
hylo :: Functor f =>
  (f b -> b) -> (a -> f a) -> a -> b
hylo f g = h where h = f . fmap h . g
\end{minted}
\noindent \hs{g} generates a structure, destructed right away using \hs{f}.
The goal of the next section is to implement this fusion automatically as an automatic program transformation.

\section{Deforestation on fixed-point types}
\label{sec:rectypes}
%!TEX root = fusion.tex

Deforestation is a process first introduced by Wadler \cite{WADLER1990231} and realized in Haskell by Gill et al. \cite{Gill:1993:SCD:165180.165214}. It is the idea that one can eliminate a structure produced by a "good function" (like an anamorphism) and destructed right away by a catamorphism, since we know how the intermediate structure was built.

A typical example on lists is the expression:
\begin{minted}{Haskell}
map g (map f xs)
\end{minted}
\hs{map} will destruct the original list and produce a new one, which will be destructed right away by the outer \hs{map}. It is easy to see that the above expression is equivalent to a more efficient one:
\begin{minted}{Haskell}
map (g . f) xs
\end{minted}
\noindent where we go through the list only once.
The rewriting operation we made in this example is again not at all dependent of the list structure: we just used the fact that we knew how the list produced by \hs{map} is made. This concept can be simply generalized to recursive structures (as theoretically explained by Johann \cite{Johann:2002:GSF:641433.641471}) and we will see practically how to make this work in the next part.

\subsection{The idea}
\label{sec:the-idea}
Deforestation can be implemented on fixed-point types imitating the work of Gill et al. \cite{Gill:1993:SCD:165180.165214} on lists. Applying \hs{cata go} to a fixed-point structure has just the effect to replace every occurrence of \hs{Fix} by \hs{go} in the structure. Consequently, the idea is to abstract structure-producing functions with respect to \hs{Fix}. This is done using \hs{buildR}.

\begin{minted}{Haskell}
type Cata f = forall b. (f b -> b) -> b

buildR :: Cata f -> Fix f
buildR g = g Fix
\end{minted}

\noindent Intuitively, the argument of \hs{buildR} is a function that  can produce something of type \hs{b}, for all type \hs{b} and given something to "perform recursion" (that is, produce a \hs{b} from a \hs{(f b)}). For example, you can use \hs{buildR} to define \hs{ana}:
\begin{minted}{Haskell}
ana :: Functor f => (b -> f b) -> b -> Fix f
ana f b = buildR
  (\u -> let c =  u . fmap c . f in c b)
\end{minted}

\subsubsection{Validity}
At first glance, we have the lovely property that:
$$cata\ go\ (buildR\ g)\ =\ g\ go$$

This is exactly what we want: the right-side of the equation \emph{does not} build an intermediate structure. The next theorem stating this equality is proved using the rank-2-types required in \hs{Cata}'s type, but as we show in Section   \ref{sec:seq}, it is totally valid only without the presence of \hs{seq} and \hs{undefined}.

\begin{theorem}
Let $F$ be a fixed parametric type and a valid instance of $Functor$, and a type $A$. Let $g$ and $go$ be functions with types:
\begin{itemize}
	\item $g :: \forall\ B.\ (F\ B \to B) \to B$
	\item $go :: F\ A \to A $
\end{itemize}
then,
$$cata\ go\ (buildR\ g)\ =\ g\ go$$
\end{theorem}
\begin{proof}
This is a consequence of free theorems of Wadler \cite{Wadler:1989:TF:99370.99404} as for the \verb|foldr/build| rule \cite{Gill:1993:SCD:165180.165214}. The free theorem associated with $g$'s type is that, for all types $C$ and $D$, $f\ ::\ C \to D$, $p\ ::\ F\ C \to C$ and $q\ ::\ F\ D \to D$, then we have the following implication:
\begin{align*}
[\ \forall& x::C.\ f\ (p\ x)\ =\ q\ (fmap\ f\ x)\ ]\\
&\implies f\ (g\ p)\ =\ g\ q
\end{align*}
Now we can instantiate (using $fix$ to denote the unique constructor of the $Fix$ type):
\begin{itemize}
	\item $C\ =\ Fix\ F$, $D\ =\ A$
	\item $f\ =\ cata\ go$
	\item $p\ =\ fix$
	\item $q\ =\ go$
\end{itemize}
then we have
\begin{align*}
&[\ \forall x::F\ (Fix\ F).\\
&cata\ go\ (fix\ x)\ =\ go\ (fmap\ (cata\ go)\ x)\ ]\\
&\implies cata\ go\ (g\ fix)\ =\ g\ go
\end{align*}
Given the definition of $cata$, the premise trivially holds, so we have the conclusion:
$$cata\ go\ (g\ fix)\ =\ g\ go$$
Given the definition of $buildR$, this is exactly what we required.
\end{proof}

\subsubsection{A note on $\bot$ and \hs{seq}}
\label{sec:seq}
Free theorems are known to (partially) fail in presence of $\bot$ (\hs{undefined} in Haskell) and \hs{seq}, defined in the Haskell report \cite{haskellReport} as a function respecting the following equation:
\begin{align*}
seq\ \bot\ b\ &=\ \bot\\
seq\ a\ b\ &=\ b\ \mathrm{,if}\ a\ \neq\ \bot
\end{align*}
\hs{seq} is a function forcing the evaluation of its first argument and returning the second. This characterization is sufficient to break our rule (as it also breaks the \verb|foldr/build| one). Consider the following example:

\begin{minted}{Haskell}
data BinF a = L | R deriving (Functor)
type Bin = Fix BinF

bad :: Int
bad = cata go $ buildR $ \u -> u L `seq` u R
  where
    go L = undefined
    go R = 0
\end{minted}
\hs{Bin} is a non-recursive type with two constructors defined in terms of \hs{Fix} (which allow us to use \hs{cata} and \hs{buildR}). Now consider the \hs{bad} expression: it constructs a \hs{Bin} using \hs{R} and then destruct it, with a partial function, defined only for \hs{R}. Without rewrite rules, the function simply returns \hs{0} (since \hs{u} is replaced by \hs{Fix}). Now with our \verb|cata/buildR| rule, \hs{u} is replaced by \hs{go}, and we precisely force the definition of \hs{go} on \hs{L}, so the whole expression in now \hs{undefined}.

Fortunately, as described by Johann and Voigtländer \cite{JV04}, one can recover free theorems by requiring more properties on the combination function (like strictness and totality), but this is beyond our scope.

\subsection{Implementation}
The implementation is pretty straightforward using GHC's rewrite rules described in the great work of Peyton Jones et al. \cite{pbr}.

A rewrite rule consists in a name and body. The body is an equality were free variables are bounded by a universal quantifier. When the left part of the equality is encountered in the code (and types match), it is replaced by the right part (more details can be found in the official GHC manual \cite{ghc:manual}).

A rewrite rule can also specify a phase constraint. GHC is performing 3 distinct phases where it tries to apply rewrite rules and it inlines functions (that is, replace a function by its definition). The so-called "phase control" is the \emph{only} mechanism that allows us to ensure when the rules are firing and when functions are inlined.

We want our rules to be compatible with those for lists already in place (the \verb|foldr/build| rule of Peyton Jones et al. \cite{pbr}). They all target the \hs{build} function (defined in \verb|GHC.Base|) that is inlined in phase 1, so we have to do all our rewriting work before, in phase 2.

First of all, imitating again the work made for lists, we will add a
\begin{minted}{Haskell}
{-# INLINE [1] buildR #-}
\end{minted}
pragma, to ensure that \hs{buildR} is not inlined too quickly. Note that using the \verb|INLINE [1]| pragma prevents the inlining in phase 2 and forces it in phase 1; this is a good idea since high-order functions like \hs{buildR} generally benefit from inlining.

Our rule will also target \hs{cata}, we consequently also need to add an \verb|INLINE| pragma to it.
\begin{minted}{Haskell}
{-# INLINE [0] cata #-}
\end{minted}

\noindent We can now safely add the rule:
\begin{minted}{Haskell}
type Cata f = forall b. (f b -> b) -> b
{-# RULES
"cata/buildR" [~1]
  forall (f::t b -> b) (g::Cata t).
    cata f (buildR g) = g f
 #-}
\end{minted}
This rule is genuinely replacing all occurrences of\\ \hs{cata f (buildR g)} by \hs{g f} \emph{when types match}, that is when the argument of \hs{buildR} are allowing fusion. The \hs{[~1]} parameter will disable the rule in the final phase, so we can rewrite un-merged functions back.

\subsubsection{The problem of inlining recursive functions}
GHC is not (at all) very keen to inline recursive functions. Since we rewrite many things, there is sometimes an expression like:
\begin{minted}{Haskell}
cata go (Fix x)
\end{minted}
ending up in the code, and will not be optimized to the equivalent
\begin{minted}{Haskell}
go x
\end{minted}
Fortunately, the work of Peyton Jones \cite{PeytonJones:2007:CSH:1291151.1291200} offers the needed optimizations, through call-pattern specialisation, enabled using the GHC's flag: \verb|-fspec-constr|.

\subsection{Rewriting unneeded destructions}
\label{sec:bind-def}
All our work if focused on destructing structures. It can have unwanted implications. Consider the following code, implementing \hs{bind} for recursive trees (as defined in Figure \ref{fig:listtree}):

\begin{minted}{Haskell}
bind :: Tree a -> (a -> Tree b) -> Tree b
bind t f = buildR $ \u ->
  cata
    (\x ->
      case x of
        EmptyF -> u EmptyF
        LeafF x -> cata u (f x)
        NodeF a b -> u (NodeF a b))
    t
\end{minted}

\noindent Given a \hs{u} of the right type, one can indeed go trough the tree until it finds a leaf, and replaces this leaf by an application of a function producing a tree (here called \hs{f}). But why do we need the unneeded \hs{cata u} after? Because one can replace \hs{u} by another function than \hs{Fix}, for example, something that calculates the number of leaves of the tree, thus we effectively need to "destruct" the tree constructed by \hs{f}.

The problem is that, without rewriting, \hs{u} will be replaced by \hs{Fix}, and we will end up with a \hs{cata Fix} (which is replacing occurrences of \hs{Fix} in the structure by \hs{Fix}, so equivalent to \hs{id}) applied to the tree produced by \hs{f}. To avoid this unneeded destruction/construction, the solution is to use a second rewrite rule that get rid of the identity catamorphism.

Due to the work of Breitner et al. \cite{Breitner:2014:SZC:2692915.2628141} we cannot target \hs{Fix} in our rewrite rule since it will be optimized away. Opportunely, this use case was planned (Section 6.5 of \cite{Breitner:2014:SZC:2692915.2628141}) and we can target the function that replace \hs{Fix}: \hs{coerce}.

\begin{minted}{Haskell}
{-# RULES
"cata/id"
  cata coerce = id
 #-}
\end{minted}

\noindent As a side note, this problem induces a possible limitation which is easily solved: we \emph{need} to enable optimizations. Rewrite rules mechanism is enabled either when required by the user or where optimization is enabled (with \verb|-O| or \verb|-O2| flags); fortunately, \verb|cabal| enforces optimizations by default. If rewrite rules are not enabled, the direct use of \hs{build} in the body of \hs{bind} will introduce an unwanted and unneeded traversal of a part of the tree as showed. This is solved (as in the \verb|GHC.Base| module), by defining "normally" \hs{bind} as a catamorphism, and adding a rule to rewrite it in its \hs{buildR} form.

Note also that, as Voigtländer \cite{Voigtlnder2008TypesFP} noticed for \hs{map}, \hs{bind} for \hs{Tree} is expressed both with \hs{buildR} and \hs{cata}, thus it can be fused with \emph{both} producers and consumers of trees.

\subsection{Examples}
\subsubsection{foldr/build}
The \verb|foldr/build| rule is a particular case of our\\ \verb|cata/buildR| rule, using:
\begin{minted}{haskell}
map :: (a -> b) -> List a -> List b
map f xs = buildR $ \u ->
  let go x = case x of
    NilF -> u NilF
    ConsF a b -> u (ConsF (f a) b)
  in cata go xs
\end{minted}

\noindent The expression:
\begin{minted}{Haskell}
foldr c e (map f xs)
\end{minted}
(with \hs{foldr} defined as in Section    \ref{subsec:defi}) is rewritten, only by inlining, as:
\begin{minted}{haskell}
cata go' $ buildR $ \u ->
  let go x = case x of
    NilF -> u NilF
    ConsF a b -> u (ConsF (f a) b)
  in cata go xs
\end{minted}

\noindent With the \verb|cata/buildR| rule, this will be rewritten to:
\begin{minted}{haskell}
let go x = case x of
  NilF -> go' NilF
  ConsF a b -> go' (ConsF (f a) b)
in cata go xs
\end{minted}
That is exactly what we wanted!

\subsubsection{Automatic hylomorphisms}
As we saw in Section \ref{sec:the-idea}, we can define \hs{ana} in term of \hs{buildR}. The fusion of a catamorphism (expressed with \hs{cata}) composed with an anamorphism (expressed with \hs{ana}, so with \hs{buildR}) is then totally automatic, and will effectively produce a hylomorphism.

\subsection{Compatibility with GHC's rules on lists}
Our work is compatible with the one already made in the \verb|GHC.Base| and \verb|GHC.List| modules (described by Peyton Jones et al. \cite{pbr}) in the sense that one can write a function that destruct a (Prelude) list (using \verb|foldr| or any "good consumer" of lists) to a fixed-point structure \emph{that can be fused with list producers and fixed-point structure consumers}.

Let us take an example with \hs{Tree}, by defining\\ \hs{treeFromList} that converts a list to a "degenerated" tree, where each node has a leaf:
\begin{minted}{Haskell}
treeFromList :: [Tree a] -> Tree a
treeFromList xs = buildR $ \u ->
  foldr
    (\x -> u . NodeF (cata u x))
    (u EmptyF)
    xs
\end{minted}
\noindent \hs{treeFromlist} is made with both \hs{foldr} and \hs{buildR}, so it can fuse with both \hs{cata} or \hs{build}.

The reverse is also possible, for example to convert a fixed-point list to a Prelude one:
\begin{minted}{Haskell}
flistToList :: List a -> [a]
flistToList xs = GHC.Base.build $ \c n ->
  let go xs = case xs of
    NilF -> n
    ConsF a b -> c a b
  in cata go xs
\end{minted}
\noindent \hs{treeFromlist} is made with both \hs{cata} and \hs{build}, so it can fuse with both \hs{buildR} or \hs{foldr}.

It is simple to see that the system composed by \verb|cata/buildR| and \verb|foldr/build| rules is \emph{confluent}, that is, if there is an expression where both rules can fire, the order is not important (because both rules are just like a $\beta$-reduction with some manipulation of arguments).

\section{Going further with paramorphisms}
\label{sec:para}
%!TEX root = fusion.tex
\subsection{Paramorphisms}

As far as we did, we worked only on catamorphisms. These are not really satisfying to define recursive functions that also use the unconsumed part of the structure. A \emph{paramorphism} (introduced by Meertens \cite{Meertens1992}) are a direct response, as it allows the combination function to have a look on the unconsumed part of the structure (imitating the full power of recursion).

They can be defined for lists using \minline{para}
\begin{minted}{Haskell}
para :: (a -> [a] -> b -> b) -> b -> [a] -> b
para _ z []     = z
para k z (x:xs) = k x xs (para k z xs)
\end{minted}

Again, we can also express paramorphisms with types defined in terms of \minline{Fix}, but using a little trick: The usual way (as in the \verb|recursion-schemes| package \cite{ekmett:eschems}) is with \minline{para}:
\begin{minted}{Haskell}
para ::
Functor f => (f (Fix f, a) -> a) -> Fix f -> a
para f = c
  where
    c (Fix x) = f (fmap (\x -> (x,p x)) x)
\end{minted}

What is happening here? As with \minline{cata}, when we are trying to destruct a recursive structure, we are doing the recursive step plus giving a copy of the original data.
For example, we can encode a classic paramorphism, which is giving the list of proper suffixes of a list:
\begin{minted}{Haskell}
suff :: List a -> List (List a)
suff = para go
  where
    go NilF = nil
    go (ConsF x (xs,b)) = cons (cons x xs) b
\end{minted}

\subsection{Fusion for paramorphisms}

Now consider this less trivial paramorphism:
\begin{minted}{Haskell}
data Hit = Miss | Tail | Edge
  deriving (Eq, Ord, Show)

has2Leaves :: Eq a => a -> a -> Tree a -> Bool
has2Leaves s t = (==) Edge . para go
  where
    go EmptyF = Miss
    go (LeafF x) = if x == s then Tail else Miss
    go (NodeF (_,a') (b,b')) =
      case a' of
        Miss -> b'
        Tail -> if hasLeaf t b then Edge else Tail
        Edge -> Edge
\end{minted}


As noticed Domínguez et al. \cite{paramorphismFusion} this is not always a good idea.

Consider this example:
% + TODO relire papier PeytonJones (playing by the rules)

\section{Further work}
\label{sec:related}
%!TEX root = fusion.tex

The work presented here can be pursued in several ways: 
\begin{itemize}
\item As mentioned by Peyton Jones et al. \cite{pbr}, forcing the inlining of functions to be able to rewrite them can produce a lot of code. A possible solution is to introduce a rule per function that rewrite them back if fusion did not happen.

\item \hs{seq} and $\bot$ does not interact well with our rules, studying their impact and how to recover semantic equivalence in this precise case would be interesting, as it was made for the \verb|foldr/build| rule and others.

\item As explained, fusion of paramorphisms is not always a good idea. It would be great to have a convenient type criterion that makes the fusion interesting from a performance point of view.

\item We studied only two rules systems, maybe others can be adapted to work on fixed-point structures.
\end{itemize}

\begin{acks}
I would like to thank Andrey Mokhov as he suggested me to start this paper and provided great counsels and constant encouragements during the writing. Yann Régis-Gianas, Jospeh Priou and Maxime Flin provided very useful comments and remarks during several presentations of this work.
\end{acks}


\bibliographystyle{ACM-Reference-Format}
\bibliography{publications}
\end{document}

