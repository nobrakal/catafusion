%!TEX root = fusion.tex

As far as we did, we worked only on catamorphisms. How can we implement paramorphisms?
The usual way (as in \cite{ekmett:eschems}) is with \minline{para}:
\begin{minted}{Haskell}
para ::
Functor f => (f (Fix f, a) -> a) -> Fix f -> a
para f = c
  where
    c (Fix x) = f (fmap (\x -> (x,p x)) x)
\end{minted}

What is happening here? As with \minline{cata}, when we are trying to destruct a recursive structure, we are doing the recursive step plus giving a copy of the original data.
For example, we can encode a classic paramorphism, where we gives the list of proper suffixes of a list:
\begin{minted}{Haskell}
suff :: List a -> List (List a)
suff = para go
  where
    go NilF = nil
    go (ConsF x (xs,b)) = cons (cons x xs) b
\end{minted}

\subsection{Fusion for paramorphisms}

Now consider this less trivial paramorphism:
\begin{minted}{Haskell}
data Hit = Miss | Tail | Edge deriving (Eq, Ord, Show)

has2Leaves :: Eq a => a -> a -> Tree a -> Bool
has2Leaves s t = (==) Edge . para go
  where
    go EmptyF = Miss
    go (LeafF x) = if x == s then Tail else Miss
    go (NodeF (_,a') (b,b')) =
      case a' of
        Miss -> b'
        Tail -> if hasLeaf t b then Edge else Tail
        Edge -> Edge
\end{minted}


As noticed Domínguez et al. \cite{paramorphismFusion} this is not always a good idea.

Consider this example:
% + TODO relire papier PeytonJones (playing by the rules)